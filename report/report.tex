\documentclass[a4paper,10pt]{article}

\usepackage[utf8]{inputenc}
\usepackage[english]{babel}
\usepackage[T1]{fontenc}
\usepackage{mathpazo} %http://www.ctan.org/tex-archive/fonts/mathpazo
\usepackage{stmaryrd} %http://www.ctan.org/pkg/stmaryrd
\usepackage{amsmath} %http://www.ctan.org/pkg/amsmath
\usepackage{amssymb}
\usepackage{mathrsfs}

\usepackage{fancyhdr}
\usepackage[a4paper]{geometry}
\geometry{hscale=0.70,vscale=0.70,centering}
%\onehalfspacing
\setlength\parindent{0pt}
\pagestyle{fancy}

\usepackage{amsthm} %http://www.ctan.org/pkg/amsthm
\usepackage{proof}

\usepackage[colorlinks=true]{hyperref} %http://www.ctan.org/tex-archive/macros/latex/contrib/hyperref/
\hypersetup{urlcolor=black,linkcolor=black}

\usepackage{footmisc} %http://www.ctan.org/tex-archive/macros/latex/contrib/footmisc

\usepackage{enumerate}
\usepackage{enumitem}
\usepackage{ulem} %http://www.ctan.org/tex-archive/macros/latex/contrib/ulem
\normalem
\usepackage{cancel} %http://www.ctan.org/tex-archive/macros/latex/contrib/cancel

%\usepackage{fullpage} %http://www.ctan.org/tex-archive/macros/latex/contrib/preprint/
\setlength{\parindent}{0pt}
\setlength{\parskip}{\medskipamount}

\usepackage{pgffor}
\usepackage{tikz}
\usetikzlibrary{arrows,shapes.arrows, chains, positioning, automata, graphs}
\usepackage{graphviz}

\usepackage[ruled,vlined,english]{algorithm2e}
\providecommand{\SetAlgoLined}{\SetLine}
\providecommand{\DontPrintSemicolon}{\dontprintsemicolon}

\usepackage{forest}
\usepackage{comment} %http://www.ctan.org/tex-archive/macros/latex/contrib/comment
\usepackage{multirow} %http://www.ctan.org/tex-archive/macros/latex/contrib/multirow
\usepackage{diagbox} %http://www.ctan.org/tex-archive/macros/latex/contrib/diagbox

\usepackage{textcomp} %http://www.ctan.org/pkg/textcomp

\usepackage{listings} %http://www.ctan.org/tex-archive/macros/latex/contrib/listings/
\lstset{numbers=left,language=Caml}

\newcounter{ThComp}
\newcounter{DefComp}

\newtheorem*{fact}{Fact}
\newtheorem*{csq}{Consequence}
\newtheorem{thm}[ThComp]{Theorem}
\newtheorem{theorem}[ThComp]{Theorem}
\newtheorem{propo}[ThComp]{Proposition}
\newtheorem{proposition}[ThComp]{Proposition}
\newtheorem{lemma}[ThComp]{Lemma}
\newtheorem*{corol}{Corollary}
\newtheorem{prop}[ThComp]{Property}
\newtheorem{property}[ThComp]{Property}
\theoremstyle{definition}
\newtheorem*{ex}{Example}
\newtheorem*{exs}{Examples}
\newtheorem{exo}{Exercise}
\newtheorem{defi}[DefComp]{Definition}
\newtheorem*{notation}{Notation}
\newtheorem{definition}[DefComp]{Definition}
\newtheorem{algo}{Algorithm}
\theoremstyle{remark}
\newtheorem*{Rq}{Remark}
\newcommand{\ra}{\rightarrow}
\newcommand{\la}{\leftarrow}


\newcommand{\RR}{\mathbb{R}}
\newcommand{\ZZ}{\mathbb{Z}}
\newcommand{\NN}{\mathbb{N}}
\newcommand{\PP}{\mathbb{P}}
\newcommand{\EE}{\mathbb{E}}
\newcommand{\IE}{\mathbb{E}}
\newcommand{\IR}{\mathbb{R}}
\newcommand{\IZ}{\mathbb{Z}}
\newcommand{\IN}{\mathbb{N}}
\newcommand{\IP}{\mathbb{P}}

\newcommand{\zo}{\set{0,1}}

\newcommand{\cA}{\mathcal{A}}
\newcommand{\cK}{\mathcal{K}}
\newcommand{\cC}{\mathcal{C}}
\newcommand{\cP}{\mathcal{P}}
\renewcommand{\bar}{\overline}
\newcommand{\Adv}{\mathrm{Adv}}

\newcommand{\cF}{\mathcal{F}}
\newcommand{\ck}{\mathcal{K}}
\newcommand{\cL}{\mathcal{L}}
\newcommand{\cN}{\mathcal{N}}
\newcommand{\cNU}{\mathcal{NU}}
\newcommand{\A}{\mathcal{A}}
\newcommand{\K}{\mathcal{K}}
\newcommand{\C}{\mathcal{C}}
\newcommand{\U}{\mathcal{U}}
\newcommand{\B}{\mathcal{B}}
\newcommand{\F}{\mathcal{F}}
\renewcommand{\L}{\mathcal{L}}
\newcommand{\N}{\mathcal{N}}

\newcommand{\ens}[1]{\left\{ #1 \right\}}
\newcommand{\set}[1]{\left\{ #1 \right\}}
\newcommand{\abs}[1]{\left| #1 \right|}
\renewcommand{\leq}{\leqslant}
\renewcommand{\geq}{\geqslant}
\renewcommand{\le}{\leqslant}
\renewcommand{\ge}{\geqslant}
\newcommand{\cplx}[1]{\mathcal O \left( #1 \right)}
\newcommand{\floor}[1]{\left \lfloor #1 \right \rfloor}
\newcommand{\ceil}[1]{\left\lceil #1 \right\rceil}
\newcommand{\brackets}[1]{\left\llbracket #1 \right\rrbracket}
\newcommand{\donne}{\rightarrow}
\newcommand{\gives}{\rightarrow}
\newcommand{\dans}{\to}
\newcommand{\booleen}{\set{0,1}^*}
\newcommand{\eps}{\varepsilon}
\renewcommand{\implies}{~\Rightarrow~}
\newcommand{\tildarrow}{\rightsquigarrow}
\newcommand{\blank}{\texttt{\char32}}
\newcommand{\trans}[1]{\xrightarrow{#1}}
\newcommand{\rules}[1]{\xrightarrow{#1}}
\newcommand{\todo}[1]{\Large\textcolor{red}{#1}\normalsize}
\renewcommand{\todo}[1]{}
\newcommand{\argmin}{\text{argmin}}
\newcommand{\rainbowdash}{\vdash}
\newcommand{\notrainbowdash}{\nvdash}
\newcommand{\rainbowDash}{\vDash}
\newcommand{\notrainbowDash}{\nvDash}
\newcommand{\Rainbowdash}{\Vdash}
\newcommand{\notRainbowdash}{\nVdash}
\newcommand{\bottom}{\bot}

%TD/TP
\newenvironment{answer}{\color{blue}}{}


%EvalPerf
\newcommand{\Var}{\text{Var}}
\newcommand{\prob}[1]{\PP\left( #1 \right)}
\newcommand{\esp}[1]{\EE\left( #1 \right)}


%SystDist
\newcommand{\Receive}{\texttt{Receive~}}
\newcommand{\Send}{\texttt{Send~}}


%Preuves
\newcommand{\betaeq}{=_\beta}
\newcommand{\betared}{\vartriangleright_\beta}
\newcommand{\parabetared}{\vartriangleright_{||\beta}}
\newcommand{\Ackermann}{\A}


%Cplx
\newcommand{\Time}{\textsc{Time}}
\newcommand{\TIME}{\textsc{Time}}

\newcommand{\dtime}{\textsc{DTime}}
\newcommand{\dTime}{\textsc{DTime}}
\newcommand{\DTime}{\textsc{DTime}}

\newcommand{\ntime}{\textsc{NTime}}
\newcommand{\nTime}{\textsc{NTime}}
\newcommand{\NTime}{\textsc{NTime}}

\renewcommand{\P}{\textsc{P}}

\newcommand{\pTime}{\textsc{PTime}}
\newcommand{\PTime}{\textsc{PTime}}

\newcommand{\NP}{\textsc{NP}}

\newcommand{\npTime}{\textsc{NPTime}}
\newcommand{\NPTime}{\textsc{NPTime}}

\newcommand{\EXP}{\textsc{Exp}}
\newcommand{\expTime}{\textsc{Exp}}
\newcommand{\ExpTime}{\textsc{Exp}}
\newcommand{\EXPTime}{\textsc{Exp}}

\newcommand{\Space}{\textsc{Space}}

\newcommand{\dSpace}{\textsc{DSpace}}
\newcommand{\DSpace}{\textsc{DSpace}}


\newcommand{\nSpace}{\textsc{NSpace}}\newcommand{\NSpace}{\textsc{NSpace}}

\newcommand{\pSpace}{\textsc{PSpace}}
\newcommand{\PSpace}{\textsc{PSpace}}

\newcommand{\npSpace}{\textsc{NPSpace}}
\newcommand{\NpSpace}{\textsc{NPSpace}}
\newcommand{\NPSpace}{\textsc{NPSpace}}

\newcommand{\SpaceTM}{\textsc{SpaceTM}}

\newcommand{\nL}{\textsc{NL}}
\newcommand{\NL}{\textsc{NL}}

\newcommand{\LL}{\textsc{L}}

\newcommand{\coNP}{co\text{-}\textsc{NP}}

\newcommand{\conL}{co\text{-}\textsc{NL}}
\newcommand{\coNL}{co\text{-}\textsc{NL}}

\newcommand{\npc}{\text{\textit{NP-C}}}

\newcommand{\PH}{\textsc{PH}}

\newcommand{\TISP}{\textsc{TISP}}

\newcommand{\Size}{\textsc{Size}}
\newcommand{\SIZE}{\textsc{Size}}


\lhead{Tom Cornebize}
\chead{\textbf{TP Cryptography}}
\rhead{28/02/2015}



\hypersetup{
    pdftitle={TP Cryptography},
    pdfauthor={Tom Cornebize}
}

\begin{document}

%%%%%%%%%%%%%%%%%%%%%%%%%%%%%%%%%%%%%%%%%%%%%%%%%%%%%%%%%%%%%%%%%%%%%%%%%%%%%%%%

\begin{enumerate}[label=\textbf{\arabic*})]
    \setlength\itemsep{2em}
    \item See source code.

    \item
        \[
        \bordermatrix{
        & 0 & 1 & 2 & 3 & 4 & 5 & 6 & 7 & 8 & 9 & 10 & 11 & 12 & 13 & 14 & 15\cr
        0 & 16 & 8 & 8 & 8 & 8 & 8 & 8 & 8 & 8 & 8 & 8 & 8 & 8 & 8 & 8 & 8\cr
        1 & 8 & 10 & 8 & 6 & 8 & 14 & 8 & 10 & 10 & 8 & 6 & 8 & 6 & 8 & 10 & 8\cr
        2 & 8 & 4 & 8 & 8 & 6 & 10 & 6 & 6 & 8 & 8 & 4 & 8 & 10 & 10 & 6 & 10\cr
        3 & 8 & 6 & 8 & 6 & 6 & 8 & 6 & 8 & 10 & 8 & 10 & 8 & 8 & 10 & 8 & 2\cr
        4 & 8 & 10 & 8 & 10 & 8 & 6 & 8 & 6 & 14 & 8 & 6 & 8 & 10 & 8 & 10 & 8\cr
        5 & 8 & 8 & 4 & 8 & 8 & 8 & 12 & 8 & 8 & 12 & 8 & 8 & 8 & 12 & 8 & 8\cr
        6 & 8 & 10 & 4 & 10 & 10 & 8 & 6 & 8 & 6 & 4 & 6 & 8 & 8 & 10 & 8 & 6\cr
        7 & 8 & 8 & 8 & 8 & 10 & 10 & 10 & 2 & 8 & 8 & 8 & 8 & 6 & 6 & 6 & 6\cr
        8 & 8 & 4 & 6 & 6 & 10 & 6 & 8 & 8 & 8 & 8 & 6 & 10 & 6 & 6 & 12 & 8\cr
        9 & 8 & 10 & 6 & 8 & 2 & 8 & 8 & 6 & 6 & 8 & 8 & 10 & 8 & 6 & 10 & 8\cr
        10 & 8 & 8 & 10 & 10 & 8 & 8 & 10 & 10 & 8 & 8 & 6 & 14 & 8 & 8 & 6 & 6\cr
        11 & 8 & 6 & 10 & 12 & 8 & 10 & 10 & 8 & 6 & 8 & 8 & 6 & 10 & 8 & 12 & 6\cr
        12 & 8 & 6 & 6 & 8 & 6 & 8 & 12 & 10 & 10 & 4 & 8 & 6 & 8 & 6 & 6 & 8\cr
        13 & 8 & 8 & 10 & 10 & 6 & 6 & 8 & 8 & 8 & 8 & 6 & 6 & 2 & 10 & 8 & 8\cr
        14 & 8 & 6 & 6 & 12 & 8 & 10 & 6 & 8 & 10 & 8 & 12 & 10 & 6 & 8 & 8 & 10\cr
        15 & 8 & 8 & 10 & 6 & 8 & 8 & 10 & 6 & 8 & 4 & 10 & 10 & 8 & 12 & 10 & 10\cr
        }
        \]
    \item
    The probability the farthest from $\frac{1}{2}$ is $p_{0, 0} = 1$.

    Then the ouples $(a, b)$ with probability $p_{a, b} = \frac{1}{2} \pm \frac{6}{16}$ are:

    $(1, 5), (3, 15), (4, 8), (7, 7), (9, 4), (10, 11), (13, 12)$

    \item
    Notice that we have $x \cdot (a \oplus b) = (x \oplus a)\cdot(x \oplus b)$

    Take $(a, b)$ in the above list. We have $p_{a, b} = \frac{1}{2} \pm \frac{6}{16}$.

    Let $A = a\underbrace{0\dots0}_{28}$ and $B = b\underbrace{0\dots0}_{28}$.

    Fix $K_0, K_1$. Take $m \hookleftarrow \cU\left(\zo^{32}\right)$

    \begin{align*}
        \prob{A\cdot m = P(B)\cdot x_1} &= \prob{A\cdot (x_0 \oplus K_0) = P(B) \cdot (P(S(x_0))\oplus K_1)} && \text{by definition}\\
            &= \prob{(A\cdot x_0) \oplus (A \cdot K_0) = (P(B) \cdot P(S(x_0)) \oplus (P(B) \cdot K_1)} && \text{by distributivity}\\
    \end{align*}

    \begin{itemize}
        \item If $A \cdot K_0 = P(B) \cdot K_1$, then:
        \begin{align*}
            \prob{A\cdot m = P(B)\cdot x_1} &= \prob{A\cdot x_0 = P(B) \cdot P(S(x_0)}\\
                &= \prob{A\cdot x_0 = B \cdot S(x_0)} && \text{since } P(x) \cdot P(y) = x \cdot y\\
                &= \frac{1}{2} \pm \frac{6}{16} && \text{by question 3}
        \end{align*}
        \item If $A \cdot K_0 \neq P(B) \cdot K_1$, then:
        \begin{align*}
            \prob{A\cdot m = P(B)\cdot x_1} &= 1-\prob{A\cdot x_0 = P(B) \cdot P(S(x_0)}\\
                &= 1-\prob{A\cdot x_0 = B \cdot S(x_0)} && \text{since } P(x) \cdot P(y) = x \cdot y\\
                &= \frac{1}{2} \pm \frac{6}{16} && \text{by question 3}
        \end{align*}
    \end{itemize}
    Thus:
    \[\prob{A\cdot m = P(B)\cdot x_1} = \frac{1}{2} \pm \frac{6}{16}\]

    We check that the experimental distribution is close to the probabilities $\frac{2}{16} = 0.125$ or $\frac{14}{16} = 0.875$.

    For each couple $(a, b)$ repeat $10$ times:
    \begin{itemize}
        \item Generate randomly and uniformly a pair of keys $(K_0, K_1)$.
        \item Generate randomly and uniformly $1000000$ messages. Compute the distribution.
    \end{itemize}
    Here are the obtained results:

    \begin{tabular}{|c|ccccc|}
        \hline
        $(1, 5)$ & $0.12518$ & $0.874367$ & $0.124699$ & $0.125047$ & $0.875031$\\
            & $0.874645$ & $0.125219$ & $0.875393$ & $0.874762$ & $0.125111$\\
        \hline
        $(3, 15)$ & $0.875216$ & $0.125296$ & $0.124507$ & $0.124807$ & $0.874673$\\
            & $0.87494$ & $0.125572$ & $0.125003$ & $0.124727$ & $0.874382$\\
        \hline
        $(4, 8)$ & $0.875446$ & $0.124601$ & $0.875135$ & $0.874933$ & $0.125213$\\
            & $0.8752$ & $0.874451$ & $0.875246$ & $0.124821$ & $0.125733$\\
            \hline
        $(7, 7)$ & $0.874797$ & $0.125353$ & $0.124807$ & $0.875382$ & $0.124545$\\
            & $0.124791$ & $0.124774$ & $0.124996$ & $0.874914$ & $0.87527$\\
            \hline
        $(9, 4)$ & $0.875354$ & $0.874981$ & $0.874474$ & $0.875149$ & $0.124819$\\
            & $0.125254$ & $0.125348$ & $0.875209$ & $0.125144$ & $0.875051$\\
            \hline
        $(10, 11)$ & $0.124614$ & $0.875057$ & $0.125055$ & $0.874729$ & $0.874722$\\
            & $0.124989$ & $0.875278$ & $0.874723$ & $0.125386$ & $0.124934$\\
            \hline
        $(13, 12)$ & $0.124509$ & $0.125587$ & $0.87511$ & $0.874666$ & $0.875681$\\
            & $0.875177$ & $0.124752$ & $0.874682$ & $0.875346$ & $0.874852$\\
        \hline
    \end{tabular}

    These results are coherent.

    \item
    \[a\cdot x = \bigoplus_i a_i \wedge x_i\]
    Thus, the value of $P(B) \cdot x_1$ is entirely determined by the bits within the active boxes.

    Therefore, we can do a brute-force attack restricted on these boxes: it decreases the number of operations.

    \item Summary table:

    \begin{tabular}{llll}
        $(a, b)$ & $P(B)$ & active boxes & indices of bits that we can set \\
        \hline
        $(1, 5)$    & $00010100000000000000000000000000$ & $2$ & $2$ to $10$\\
        $(3, 15)$   & $00111100000000000000000000000000$ & $2$ & $2$ to $10$\\
        $(4, 8)$    & $00100000000000000000000000000000$ & $1$ & $2$ to $6$\\
        $(7, 7)$    & $00011100000000000000000000000000$ & $2$ & $2$ to $10$\\
        $(9, 4)$    & $00010000000000000000000000000000$ & $1$ & $2$ to $6$\\
        $(10, 11)$  & $00101100000000000000000000000000$ & $2$ & $2$ to $10$\\
        $(13, 12)$  & $00110000000000000000000000000000$ & $1$ & $2$ to $6$\\
    \end{tabular}

    $x_1 = P(S(m \oplus K_0))\oplus K_1$

    \item See source code.

    \item See source code.

    We found the key:
    \[K_2 = 2619021732\]

    \item See source code.

    We found the keys:
    \[\begin{cases}
        K  &= 208609208\\
        K0 &= 2377528524\\
        K1 &= 597825287\\
        K2 &= 2619021732
    \end{cases}\]
\end{enumerate}

\end{document}
