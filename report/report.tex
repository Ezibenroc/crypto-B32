\input{macros_TD}

\lhead{Tom Cornebize}
\chead{\textbf{TP Cryptography}}
\rhead{28/02/2015}



\hypersetup{
    pdftitle={TP Cryptography},
    pdfauthor={Tom Cornebize}
}

\begin{document}

%%%%%%%%%%%%%%%%%%%%%%%%%%%%%%%%%%%%%%%%%%%%%%%%%%%%%%%%%%%%%%%%%%%%%%%%%%%%%%%%

\begin{enumerate}[label=\textbf{\arabic*})]
    \setlength\itemsep{2em}
    \item See source code.

    \item
        \[
        \bordermatrix{
        & 0 & 1 & 2 & 3 & 4 & 5 & 6 & 7 & 8 & 9 & 10 & 11 & 12 & 13 & 14 & 15\cr
        0 & 16 & 8 & 8 & 8 & 8 & 8 & 8 & 8 & 8 & 8 & 8 & 8 & 8 & 8 & 8 & 8\cr
        1 & 8 & 10 & 8 & 6 & 8 & 14 & 8 & 10 & 10 & 8 & 6 & 8 & 6 & 8 & 10 & 8\cr
        2 & 8 & 4 & 8 & 8 & 6 & 10 & 6 & 6 & 8 & 8 & 4 & 8 & 10 & 10 & 6 & 10\cr
        3 & 8 & 6 & 8 & 6 & 6 & 8 & 6 & 8 & 10 & 8 & 10 & 8 & 8 & 10 & 8 & 2\cr
        4 & 8 & 10 & 8 & 10 & 8 & 6 & 8 & 6 & 14 & 8 & 6 & 8 & 10 & 8 & 10 & 8\cr
        5 & 8 & 8 & 4 & 8 & 8 & 8 & 12 & 8 & 8 & 12 & 8 & 8 & 8 & 12 & 8 & 8\cr
        6 & 8 & 10 & 4 & 10 & 10 & 8 & 6 & 8 & 6 & 4 & 6 & 8 & 8 & 10 & 8 & 6\cr
        7 & 8 & 8 & 8 & 8 & 10 & 10 & 10 & 2 & 8 & 8 & 8 & 8 & 6 & 6 & 6 & 6\cr
        8 & 8 & 4 & 6 & 6 & 10 & 6 & 8 & 8 & 8 & 8 & 6 & 10 & 6 & 6 & 12 & 8\cr
        9 & 8 & 10 & 6 & 8 & 2 & 8 & 8 & 6 & 6 & 8 & 8 & 10 & 8 & 6 & 10 & 8\cr
        10 & 8 & 8 & 10 & 10 & 8 & 8 & 10 & 10 & 8 & 8 & 6 & 14 & 8 & 8 & 6 & 6\cr
        11 & 8 & 6 & 10 & 12 & 8 & 10 & 10 & 8 & 6 & 8 & 8 & 6 & 10 & 8 & 12 & 6\cr
        12 & 8 & 6 & 6 & 8 & 6 & 8 & 12 & 10 & 10 & 4 & 8 & 6 & 8 & 6 & 6 & 8\cr
        13 & 8 & 8 & 10 & 10 & 6 & 6 & 8 & 8 & 8 & 8 & 6 & 6 & 2 & 10 & 8 & 8\cr
        14 & 8 & 6 & 6 & 12 & 8 & 10 & 6 & 8 & 10 & 8 & 12 & 10 & 6 & 8 & 8 & 10\cr
        15 & 8 & 8 & 10 & 6 & 8 & 8 & 10 & 6 & 8 & 4 & 10 & 10 & 8 & 12 & 10 & 10\cr
        }
        \]
    \item
    The probability the farthest from $\frac{1}{2}$ is $p_{0, 0} = 1$.

    Then the ouples $(a, b)$ with probability $p_{a, b} = \frac{1}{2} \pm \frac{6}{16}$ are:

    $(1, 5), (3, 15), (4, 8), (7, 7), (9, 4), (10, 11), (13, 12)$

    \item
    Notice that we have $x \cdot (a \oplus b) = (x \oplus a)\cdot(x \oplus b)$
    Take $(a, b) = (13, 12) = (1101_2, 1100_2)$. We have $p_{13, 12} = 2$.

    Let $A = 1101\underbrace{0\dots0}_{28}$ and $B = 1100\underbrace{0\dots0}_{28}$.

    Fix $K_0, K_1$. Take $m \hookleftarrow \cU\left(\zo^{32}\right)$

    \begin{align*}
        \prob{A\cdot m = P(B)\cdot x_1} &= \prob{A\cdot (x_0 \oplus K_0) = P(B) \cdot (P(S(x_0))\oplus K_1)} && \text{by definition}\\
            &= \prob{(A\cdot x_0) \oplus (A \cdot K_0) = (P(B) \cdot P(S(x_0)) \oplus (P(B) \cdot K_1)} && \text{by distributivity}\\
    \end{align*}

    \begin{itemize}
        \item If $A \cdot K_0 = P(B) \cdot K_1$, then:
        \begin{align*}
            \prob{A\cdot m = P(B)\cdot x_1} &= \prob{A\cdot x_0 = P(B) \cdot P(S(x_0)}\\
                &= \prob{A\cdot x_0 = B \cdot S(x_0)} && \text{since } P(x) \cdot P(y) = x \cdot y\\
                &= \frac{2}{16} && \text{by question 3}
        \end{align*}
        \item If $A \cdot K_0 \neq P(B) \cdot K_1$, then:
        \begin{align*}
            \prob{A\cdot m = P(B)\cdot x_1} &= 1-\prob{A\cdot x_0 = P(B) \cdot P(S(x_0)}\\
                &= 1-\prob{A\cdot x_0 = B \cdot S(x_0)} && \text{since } P(x) \cdot P(y) = x \cdot y\\
                &= \frac{14}{16} && \text{by question 3}
        \end{align*}
    \end{itemize}
    Thus:
    \[\prob{A\cdot m = P(B)\cdot x_1} = \frac{1}{2} \pm \frac{6}{16}\]

    We check that the experimental distribution is close to the properties $\frac{2}{16} = 0.125$ or $\frac{14}{16} = 0.875$.

    For each couple $(a, b)$ repeat $10$ times:
    \begin{itemize}
        \item Generate randomly and uniformly a pair of keys $(K_0, K_1)$.
        \item Generate randomly and uniformly $1000000$ messages. Compute the distribution.
    \end{itemize}
    Here are the obtained results:

    \begin{tabular}{|c|ccccc|}
        \hline
        $(1, 5)$ & $0.12518$ & $0.874367$ & $0.124699$ & $0.125047$ & $0.875031$\\
            & $0.874645$ & $0.125219$ & $0.875393$ & $0.874762$ & $0.125111$\\
        \hline
        $(3, 15)$ & $0.875216$ & $0.125296$ & $0.124507$ & $0.124807$ & $0.874673$\\
            & $0.87494$ & $0.125572$ & $0.125003$ & $0.124727$ & $0.874382$\\
        \hline
        $(4, 8)$ & $0.875446$ & $0.124601$ & $0.875135$ & $0.874933$ & $0.125213$\\
            & $0.8752$ & $0.874451$ & $0.875246$ & $0.124821$ & $0.125733$\\
            \hline
        $(7, 7)$ & $0.874797$ & $0.125353$ & $0.124807$ & $0.875382$ & $0.124545$\\
            & $0.124791$ & $0.124774$ & $0.124996$ & $0.874914$ & $0.87527$\\
            \hline
        $(9, 4)$ & $0.875354$ & $0.874981$ & $0.874474$ & $0.875149$ & $0.124819$\\
            & $0.125254$ & $0.125348$ & $0.875209$ & $0.125144$ & $0.875051$\\
            \hline
        $(10, 11)$ & $0.124614$ & $0.875057$ & $0.125055$ & $0.874729$ & $0.874722$\\
            & $0.124989$ & $0.875278$ & $0.874723$ & $0.125386$ & $0.124934$\\
            \hline
        $(13, 12)$ & $0.124509$ & $0.125587$ & $0.87511$ & $0.874666$ & $0.875681$\\
            & $0.875177$ & $0.124752$ & $0.874682$ & $0.875346$ & $0.874852$\\
        \hline
    \end{tabular}

    These results are coherent.

    \item \todo{TODO}

    \item Tabular:

    \begin{tabular}{lll}
        $(a, b)$ & $P(B)$ & active boxes \\
        \hline
        $(1, 5)$    & $00010100000000000000000000000000$ & $2$\\
        $(3, 15)$   & $00111100000000000000000000000000$ & $2$\\
        $(4, 8)$    & $00100000000000000000000000000000$ & $1$\\
        $(7, 7)$    & $00011100000000000000000000000000$ & $2$\\
        $(9, 4)$    & $00010000000000000000000000000000$ & $1$\\
        $(10, 11)$  & $00101100000000000000000000000000$ & $2$\\
        $(13, 12)$  & $00110000000000000000000000000000$ & $1$\\
    \end{tabular}

    $x_1 = P(S(m \oplus K_0))\oplus K_1$
\end{enumerate}

\end{document}
